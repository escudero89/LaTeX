\documentclass[10pt,a4paper]{article}
\usepackage[utf8]{inputenc}
\usepackage[margin=1in]{geometry}
\thispagestyle{empty}

\usepackage{amsmath}
\usepackage{amsfonts}
\usepackage{amssymb}

\usepackage{parskip}

\usepackage{listings}
\usepackage{xcolor}

\usepackage{enumerate}

\usepackage{hyperref}

\usepackage{float}
\usepackage[font=small,labelfont=bf]{caption}
\usepackage{wrapfig}

\usepackage{graphicx}
\restylefloat{figure}

\usepackage{cancel}

\usepackage{multicol}
\setlength{\columnsep}{22pt}

\usepackage{colortbl}

\author{Cristian Escudero}
\title{Resumen II\\Inteligencia Computacional}
\begin{document}

\section*{Lógica borrosa 2 (Unidad V)}
\begin{quote}
Memorias asociativas borrosas como mapeos, reglas borrosas simples y compuestas, ejemplos. Codificación de reglas borrosas: discretización, memorias asociativas borrosas hebbianas, codificaciones por correlación-mínimo y correlación-producto, bidireccionalidad. Composición de reglas. Métodos de máximo y centroide borroso. Inferencia de Takagi-Sugeno-Kang. Conjuntos de membresía continuos, representación y composición de varios antecedentes por consecuente.
\end{quote}

\section*{Inteligencia colectiva 1 (Unidad VI)}
\begin{quote}
Formulación de problemas de búsqueda. Estrategias de búsquedas informadas y no informadas. Complejidad temporal y espacial, completitud y optimalidad. Resolución de problemas mediante planeamiento. Lenguaje para problemas de planificación. Planificación de orden parcial. Métodos evolutivos: inspiración biológica, estructura, representación del problema, función de aptitud, mecanismos de selección, operadores elementales de variación y reproducción. Variantes de la computación evolutiva: algoritmos genéticos, programación genética, estrategias de evolución. Algoritmos multiobjetivo.
\end{quote}

\section*{Inteligencia colectiva 2 (Unidad VII))}
\begin{quote}
Autómatas de estados finitos y autómatas celulares. Agentes inteligentes. Inspiración biológica de los métodos de inteligencia colectiva. Modelos de vida artificial: comportamiento emergente, autoorganización. Colonias de hormigas: representación del problema, feromonas, búsqueda de alimento, modelo estocástico, experimento de los dos puentes. Enjambre de partículas: representación del problema, restricciones, tamaño de partícula, inicialización, ecuaciones de movimiento, distribuciones de proximidad, topología de las poblaciones.
\end{quote}

\section*{Aplicaciones (Unidad VIII)}
\begin{quote}
Configuración del problema y aplicación de las técnicas de inteligencia computacional en: clasificación de patrones, agrupación de patrones, aproximación de funciones, optimización, búsqueda de soluciones, regresión, predicción de series temporales, control de procesos, identificación de sistemas, compresión de señales, memorias y recuperación de información. Interrelaciones entre las técnicas de inteligencia computacional: sistemas híbridos.
\end{quote}

\pagebreak
\maketitle

\section{Inteligencia colectiva 1 (Unidad VI)}
\begin{quote}
Formulación de problemas de búsqueda. Estrategias de búsquedas informadas y no informadas. Complejidad temporal y espacial, completitud y optimalidad. Resolución de problemas mediante planeamiento. Lenguaje para problemas de planificación. Planificación de orden parcial. Métodos evolutivos: inspiración biológica, estructura, representación del problema, función de aptitud, mecanismos de selección, operadores elementales de variación y reproducción. Variantes de la computación evolutiva: algoritmos genéticos, programación genética, estrategias de evolución. Algoritmos multiobjetivo.
\end{quote}

\subsection{Nociones Básicas}
\begin{description}
\item \textbf{Agentes reactivos.} Basan sus acciones en una aplicación directa desde los estados a las acciones. No funcionan bien en entornos en los que esta aplicación sea demasiado grande para almacenarla y que tarde mucho en aprenderla.
\item \textbf{Agentes basados en objetivos.} Pueden tener éxito considerando las acciones futuras y lo deseable de sus resultados.
\item \textbf{Agente resolvedor de problemas.} Está basado en el anterior; deciden qué hacer para encontrar \textit{secuencias de acciones} que conduzcan a los \textit{estados deseables}.
\item \textbf{Algoritmos no informados.} No dan información sobre el problema salvo su definición.
\end{description}

\subsection{Agentes resolvedores de problemas}

El primer paso para solucionar un problema es la \textbf{formulación de un objetivo}, basado en la situación actual y la medida de rendimiento del agente. Un \textbf{objetivo} es un conjunto de estados del mundo (estados en los cuales el objetivo se encuentra satisfecho.


\pagebreak
\section{Inteligencia colectiva 2 (Unidad VII))}
\begin{quote}
Autómatas de estados finitos y autómatas celulares. Agentes inteligentes. Inspiración biológica de los métodos de inteligencia colectiva. Modelos de vida artificial: comportamiento emergente, autoorganización. Colonias de hormigas: representación del problema, feromonas, búsqueda de alimento, modelo estocástico, experimento de los dos puentes. Enjambre de partículas: representación del problema, restricciones, tamaño de partícula, inicialización, ecuaciones de movimiento, distribuciones de proximidad, topología de las poblaciones.
\end{quote}

\subsection{Autómatas de estados finitos y autómatas celulares}

Los autómatas celulares son sistemas dinámicos discretos cuyos elementos tienen una interacción constante entre sí tanto en el espacio como en el tiempo.  Tienen la capacidad de representar comportamientos complejos a partir de una dinámica sencilla. 

\subsection{Agentes inteligentes}

\subsection{Inspiración biológica de los métodos de inteligencia colectiva}
El intento inicial del concepto de enjambres de partículas fue el de simular gráficamente la grácil e impredecible coreografía de una bandada de aves, cuyos miembros vuelan de forma sincrónica y en óptima formación.



Modelos de vida artificial: comportamiento emergente, autoorganización. Colonias de hormigas: representación del problema, feromonas, búsqueda de alimento, modelo estocástico, experimento de los dos puentes. Enjambre de partículas: representación del problema, restricciones, tamaño de partícula, inicialización, ecuaciones de movimiento, distribuciones de proximidad, topología de las poblaciones.

\end{document}